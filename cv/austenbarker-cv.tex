%%%%%%%%%%%%%%%%%%%%%%%%%%%%%%%%%%%%%%%%%
% Medium Length Graduate Curriculum Vitae
% LaTeX Template
% Version 1.1 (9/12/12)
%
% This template has been downloaded from:
% http://www.LaTeXTemplates.com
%
% Original author:
% Rensselaer Polytechnic Institute (http://www.rpi.edu/dept/arc/training/latex/resumes/)
%
% Important note:
% This template requires the res.cls file to be in the same directory as the
% .tex file. The res.cls file provides the resume style used for structuring the
% document.
%
%%%%%%%%%%%%%%%%%%%%%%%%%%%%%%%%%%%%%%%%%

%----------------------------------------------------------------------------------------
%	PACKAGES AND OTHER DOCUMENT CONFIGURATIONS
%----------------------------------------------------------------------------------------

\documentclass[margin, 10pt]{res} % Use the res.cls style, the font size can be changed to 11pt or 12pt here

\usepackage{helvet} % Default font is the helvetica postscript font
%\usepackage{newcent} % To change the default font to the new century schoolbook postscript font uncomment this line and comment the one above



\usepackage{hyperref}
\hypersetup{
    colorlinks=true,
    linkcolor=blue,
    filecolor=magenta,      
    urlcolor=cyan,
}
\urlstyle{same}

\usepackage{etaremune}

\setlength{\textwidth}{5.1in} % Text width of the document


\begin{document}

%----------------------------------------------------------------------------------------
%	NAME AND ADDRESS SECTION
%----------------------------------------------------------------------------------------

\moveleft.5\hoffset\centerline{\large\bf Austen Thomas Barker} % Your name at the top
 
\moveleft\hoffset\vbox{\hrule width\resumewidth height 1pt}\smallskip % Horizontal line after name; adjust line thickness by changing the '1pt'
 
%\moveleft.5\hoffset\centerline{(\href{http://biaslab.org}{BIASlab})} % Your address
\moveleft.5\hoffset\centerline{133 Myrtle Street Santa Cruz, CA 95060}
\moveleft.5\hoffset\centerline{+1 (650)575-7253}
\moveleft.5\hoffset\centerline{austenbarker@yahoo.com, atbarker@ucsc.edu}

%----------------------------------------------------------------------------------------

\begin{resume}

\section{RESEARCH \\ INTERESTS}
 
Computer security, steganography, operating systems, kernel development, deniable systems, operating systems, flash storage and nonvolatile memory, storage systems, and cryptography.
 
\section{ACADEMIC \\ BACKGROUND}

{\sl Ph.D. Computer Science} \hfill 2018-2022 \\
\href{https://www.ucsc.edu/}{University of California, Santa Cruz } (UCSC), Santa Cruz, California
\begin{itemize}
\item Ph.D. research in computer science focusing on storage systems and security under the 
direction of Professor Darrell D. E. Long.
\item Dissertation: \emph{Artifice: A Design for Usable Deniable Storage Informed by Adversary Threat}
\end{itemize} 

{\sl M.S. Computer Science} \hfill 2017-2018 \\
\href{http://www.ucsc.edu}{University of California, Santa Cruz}, Santa Cruz, California
\begin{itemize}
\item Focused on storage and security.
\item Masters Project: \emph{Artifice: A Deniable Steganographic Storage System}
\end{itemize} 

{\sl B.S. Computer Science} \hfil 2013-2017 \\
\href{http://www.ucsc.edu}{University of California, Santa Cruz}, Santa Cruz, California
%\begin{itemize}
%\end{itemize}


\section{EMPLOYMENT \\HISTORY}

{\sl Graduate Student Researcher} \hfill March 2018 - June 2022 \\
\href{http://www.ucsc.edu}{University of California, Santa Cruz}, \href{https://ssrc.ucsc.edu}{Storage Systems Research Center}, Santa Cruz, California
\begin{itemize}
\item Graduate Student Researcher in the UCSC Storage Systems Research Center (SSRC/CRSS).
\item Currently working on \href{https://www.ssrc.ucsc.edu/proj/Artifice.html}{Artifice}, a deniable steganographic storage system and \href{https://www.ssrc.ucsc.edu/proj/securefs.html}{Lethe}, an efficient 
cryptographic secure deletions system designed for use with file systems or key-value stores. 
\end{itemize}

{\sl Teaching Assistant} \hfill January - March 2018, September - December 2019 \\
\href{http://www.ucsc.edu}{University of California, Santa Cruz}, \href{https://ssrc.ucsc.edu}{Storage Systems Research Center}, Santa Cruz, California
\begin{itemize}
\item TA for the penultimate offering of UCSC's Introduction to Operating Systems, \href{https://courses.soe.ucsc.edu/courses/cmps111/}{CMPS-111}, 
in the winter quarter of 2018. 30 students.
\item Assisted in developing curriculum and TA'd for the first offering of a new lower division class, Computer Systems and C Programming, 
\href{https://courses.soe.ucsc.edu/courses/cse13s}{CSE-13S}, in the fall quarter of 2019. 35 students.
\end{itemize}

{\sl Information Security Intern} \hfill July - December 2017, June - September 2018\\
\href{http://www.datastax.com}{DataStax}, Santa Clara, California
\begin{itemize}
\item Designed and implemented a prototype single sign-on authentication API, database interfaces (AWS RDS and Apache Cassandra), and password handling utilities in Go for a SAAS database platform.
\item Secure cloud systems setup and hardening automation. Security focused VPC log analytics API and web visualizations. 
\end{itemize}

{\sl Software Engineering Intern} \hfill June - September 2016 \\
\href{https://www.tidalscale.com}{TidalScale Inc.}, Los Gatos, California
\begin{itemize}
\item Software engineering summer internship. Also worked over the 2016 winter break between school terms. 
\item Ubuntu certification testing for a software defined server appliance.
\item Wrote a Python administration and monitoring utility for a distributed hypervisor. Collected metrics from multiple machines in a distributed system, stored them in a small time series database, and displayed using an ncurses interface.
\end{itemize}

{\sl Software Engineering Intern} \hfill June - September 2014, June - September 2015 \\
Immediate Insight, Los Altos, California
\begin{itemize}
\item Supported the deployment of an IT data analytics tool within Kaiser Permanente's IT infrastructure.
\item Installation, capacity, and operational testing of an IT data analytics platform built using Node.JS and Elasticsearch. 
\end{itemize}

\section{SPECIAL \\ ACHIEVEMENTS} 
{\sl Awards}
\begin{itemize}
\item \emph{Eagle Scout} Boy Scouts of America, Troop 30, Los Altos, California, 2012
\end{itemize}
{\sl Grants}
\begin{itemize}
\item \emph{NSF Award} Artifice research funded by NSF award \#1814347 CSR Small: A Multi-layered Deniable Steganographic File System under P.I. Prof. Darrell D. E. Long.
\end{itemize}

\section{SKILLS}
{\sl Programming Languages}\\
C, C++, Python, Go, Java, LaTeX, and Lisp/Scheme.

{\sl Technologies and Systems Proficencies}\\
FreeBSD, Linux, Windows, Git, Docker

%\bigskip
%\moveleft 1.3\hoffset\centerline{\large\bf Activities at Eindhoven Univ. of Technology (TU/e)} 

\section{PUBLICATIONS}
\begin{etaremune}
\item Kyle Fredrickson, Austen Barker, Darrell D. E. Long, "A Multiple Snapshot Attack on Deniable Storage Systems," \emph{Proceedings of the 29th International Symposium on Modeling, Analysis, and Simulation of Computer and 
Telecommunication Systems} (MASCOTS '21), November 2021, pp. 1-8.
\item Austen Barker, Yash Gupta, James Hughes, Ethan L. Miller, Darrell D. E. Long, "Rethinking the Adversary and Operational Characteristics of Deniable Storage," \emph{Journal of Surveillance, Security, and Safety} (JSSS), 2021;2;42-65.
\item Austen Barker, Yash Gupta, Sabrina Au, Eugene Chou, Ethan L. Miller, Darrell D. E. Long, "Artifice: Data in Disguise," \emph{Proceedings of the 36th International Conference on Massive Storage Systems and Technology} (MSST '20), October 2020.
\item Austen Barker, Staunton Sample, Yash Gupta, Ana McTaggart, Ethan L. Miller, Darrell D. E. Long, "Artifice: A Deniable Steganographic File System," \emph{Proceedings of the 9th USENIX Workshop on Free and Open Communications on the Internet} (FOCI '19), August 2019.
\end{etaremune}
\end{resume}
\end{document}
